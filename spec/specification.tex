\documentclass{article}

\usepackage[margin=1.0in]{geometry}
\usepackage[utf8]{inputenc}
\usepackage{csquotes}
\usepackage{t1enc}
\usepackage[english]{babel}
\usepackage{algorithm}
\usepackage{algorithmic}
\usepackage{changepage}
\usepackage{amsmath}
\usepackage{amssymb}
\usepackage{xcolor}
\usepackage[
    backend=biber,
    style=ieee
]{biblatex}
\usepackage{graphicx}
\graphicspath{ {./img/} }
\usepackage{siunitx}
\usepackage{hyperref}
\hypersetup{pdftex,colorlinks=true,allcolors=black}

\begin{document}
\title{Kotlin for Android Development Assignment}
\author{Daniel Meszaros}
\date{2020/21/1}
%\maketitle

\emph{Surveyor} is a GPS track recorder application that makes it
possible to save and share hiking routes.

Users can review these recordings, overlaying them on an interactive map,
and even export them into standard GPX files for interchange with other devices
and software.
Saved recordings can be given a custom title, so that the users may find them
more quickly when searching.

The built-in map can be used for navigation while the user is currently
recording a route, with the recorded data being overlayed onto the map.
A compass arrow is drawn at user's location indicating the direction they face.

Since there is no way to charge the phone on the trail (aside from external
battery packs) the application should be power efficient and not waste energy.
Alternative trackers like \emph{OSMtracker} and \emph{ViewRanger} stop recording
any location data when the screen is locked.
This means that the user needs to keep the screen on, which quickly drains the
battery.
An important feature of this application that it works even when it's in the
background.

Users can mark points-of-interest with the press of a button. These
markers/waypoints are stored in the recording.

The application can generate speed and elevation graphs, as well as statistical
information like average speed, top speed, etc. when reviewing a past recording.

The user are not required to login or be online to record routes, but loading
maps still need an internet connection.

I intend to use the OpenStreetMaps Android
library\footnote{https://github.com/osmdroid/osmdroid} as it's much more
detailed when it comes navigation on hiking trails.

\end{document}
